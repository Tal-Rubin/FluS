\section{Introduction}

Fluids are ubiquitous in physical systems of all scales: from astronomical-scale
Magnetohydrodynamics, planetary weather models, rivers and pipe-flows, to the smallest
blood-vessels. Fluids can describe liquids, gases and plasmas, and appear in the limit
of short mean-free-path of an \textit{N}-body problem.\\

In physics, fluids are usually described by a set of partial differential equations,
which are specified according the model we choose. In the frame of an \textit{N}-body
problem, we can apply different models to study different regimes. First of all, the
Navier-Stokes equations predict the motion of Newtonian Fluids. They are composed of
the conservations of particle, momentum, and energy, and sometimes accompanied by an
equation of state. If we consider conductive fluid subeject to the electromagnetic
field such as plasma, then the set of N-S equations must be associated with Maxwell's
equations of electromagnetism to account for the reciprocation of the fluid and the
field. Furthermore, when we consider the anisotropic transport caused by the gyromotion
of magnetized plasma, then we will need to apply Braginskii's equations. The given
initial conditions are often dicontinuous to the 1st order in the study of shocks and
rarefaction waves with these models, which makes a Riemann problem in the classical
system.\\

In this work, we want to build an \textit{N}-body solver for a set of hyperbolic
advection-diffusion PDEs. To begin with, we have a preprocessor to generate the mesh
and prescribe the initial conditions. Then we can set the boundary condition and select
a model that applies to the problem. A robust solver is developped to accostom to the
different models. And the results will be plotted by the postprocessor. This solver
will be optimized to describe the fluids in different physics contexts in the frame of
\textit{N}-body problem.\\
