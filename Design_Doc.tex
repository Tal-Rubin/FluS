\documentclass[aip, amsmath, amssymb, preprint,floatfix]{revtex4-2}
\usepackage[utf8]{inputenc}
\usepackage{graphicx} % Required for including pictures

\usepackage[hidelinks]{hyperref}
\usepackage{xcolor}
\bibliographystyle{apsrev4-2}
\include{shortcuts}

\begin{document}
\title{Design Document\\APC 524}
\date{2021}

\author{Dingyun Liu, Bingjia Yang, Kehan Cai, and Tal Rubin}
\maketitle

\section{Introduciton}

Fluids are ubiquitous in physical systems of all sizes; astronomical-scale Magnetohydrodynamics, planetary weather models, rivers and pipe-flows, to the smallest blood-vessels. Fluids can describe liquids, gases and plasmas, and appear in the limit of short mean-free-path of a \textit{N}-body problem.

\section{Mathematical Model}

The mathematical description of fluids boils down to conservation laws. 

Conservation of mass, or particle number density is called the continuity equation, and is written in Eulerian form as,
\begin{gather}
	\pdv{n}{t} + \nabla \cdot (n \bv) = 0,
\end{gather}
with $n$ being the number density, $t$, the time, and $\bv$ the velocity vector.

Conservation of the vector linear momentum is expressed by,
\begin{gather}
    \pdv{\bP}{t} + \nabla \cdot\left(\bv \bP \right) +\nabla \cdot \pi + \nabla p=  F.~\label{eq:momentum}
\end{gather}

\section{UML Diagram}

\section{Milestones}






\end{document}