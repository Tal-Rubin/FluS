\section{Proposed work}

We propose to build a code capable of handling different fluid models, and different solvers.

The code would be split into three computational steps: 
\noindent
\begin{enumerate}
\item Pre-processor: \newline
Meshing, Initial conditions.
\item Solver: \newline
Time-stepping, Boundary conditions.
\item Post-processor: \newline
Visualization.
\end{enumerate}

A driver code would navigate between the code steps. The user would define a domain and the mesh parameters. The domain dimensionality would inform the fluid model selection (possibly by reducing the number of equations from 3D to the appropriate dimentionality).

It is likely we would attempt to use an external library for the mesher - as this is a complex task on its own. 


The user would define boundary conditions, and the fluid would be initilized with a quiesent initial conditions.

We would want to have a choice between ``regular" time-stepping algorithms such as Runge-Kutta or Adams-Moulton, and symplectic algorithms such as leap-frog, or a splitting method.

The user would also determine the time-limit for the integration and the frequency of data output. The solver would advanve the fluid in time using the selected time-stepping algorithm, and save the information.

The post-processor would visualize the resulting information. The post processor would be written in python.


We plan on using GitHub using a centralized workflow, in which the master branch is updated via pull requests. Current repository is \href{git@github.com:Tal-Rubin/APC524.git}{git@github.com:Tal-Rubin/APC524.git}.


