\section{Mathematical Model}

The mathematical description of fluids boils down to conservation laws. 

Conservation of mass, or particle number density is called the continuity equation, and is written in Eulerian form as,
\begin{gather}
	\pdv{n}{t} + \nabla \cdot (n \bv) = 0,
\end{gather}
with $n$ being the number density, $t$, the time, and $\bv$ the velocity vector.

Conservation of the vector linear momentum is expressed by,
\begin{gather}
    \pdv{\bP}{t} + \nabla \cdot\left(\bv \bP + p \bI \right) +\nabla \cdot \pi  =  \bF,~\label{eq:momentum}
\end{gather}
with $\bP = m n \bv$ being the linear momentum vector, $\bv \bP$ is a dyadic tensor, obtained by a tensor product of the velocity and momentum, $p$ is the scalar pressure, $\bI$ is the unit tensor. $\pi$ is the viscous stress tensor, and $\bF$ is the body force acting on a fluid element.

Possible body forces are Lorentz force for a fluid plasma, friction force for a multi-fluid system, gravity in the presence of a gravitational field, .